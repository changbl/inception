\section{Running Orchestrator}
The \verb!orchestrator! command, located in the bin directory under the source that was fetched from github, 
is used to start and stop an inception cloud. 
The inception cloud environment consists of at least 4 Inception Control VMs (ICVMs)  in the environment:

%&indent
\begin{itemize}
\item A gateway machine that will be given a public IP address and function much in the same way as the 
boot-up VM being used to start the cloud. 

\item A controller machine used to run the OpenStack software and to provide the OpenStack Dashboard interface. 

\item A chef machine used to manage Chef installation and configuration scripts. 

\item One or more worker machines used to host the inception virtual machines (iVMs). 
\end{itemize}

\noindent
The following sections describe how orchestrator is used. 

\begin{figure}
  \centering
  \includegraphics[width=\linewidth]{start_overview.eps}
  \tiny
  \caption{\small The environment, after orchestrator has created ICVMs, showing the communication path between orchestrator and the ICVMs.
    \label{fg_overview} \normalsize
  }
  \normalsize
\end{figure}

\subsection{Starting The Inception Cloud}
The \verb!orchestrator! command is located in the \verb!bin! directory within the source cloned from github. 
The bin directory can be added to the path, or the command can be executed with a fully qualified path.  
It will probably be necessary to add the top level inception directory to the \verb!PYTHONPATH! environment variable
if inception was not installed. 
%.sp
Using the \verb!--help! option will cause all of the possible command line options to be written to the tty device. 
For the most part, at least for the first time user, only a few are needed and are described below. 

\dlbeg{0.85in}
\dlitem{-p prefix}{ 
	This command line flag is required and supplies the prefix string that is used when defining the ICVM names. 
}

\vspace{5pt}
\dlitem{-n n}{ 
	Specifies the number of work ICVMs that are created.  The iVMs which are created in the inception cloud are hosted 
	on the work VMs thus the number needed is directly related to the number of iVMs that will be created in the inception cloud. 
}

\vspace{5pt}
\dlitem{-~-image=}{ 
	Supplied the image name to be used for all ICVMs.  If not supplied a base image of Ubuntu 12.04 (64 bit) is 
	created and used for each ICVM. 
}

\vspace{5pt}
\dlitem{-~-ssh\_keyfile=}{ 
	Provides the name of the private key that is to be injected as the user key for each of the 
	control VMs that are created.
}

\vspace{5pt}
\dlitem{-~-user=}{ 
	The user name created on each node with sudo capabilities. If not given, ubuntu is used. 
}
\dlend

\noindent
The following illustrates the command to start an inception cloud with one worker and a prefix of \verb!scooter0.!

\small\begin{verbatim}
   orchestrator  -n 1 -p scooter0 \
       --ssh_keyfile=$HOME/.vmkeys/agave.shared.pk
\end{verbatim}\normalsize
\noindent
The creation and initialisation of the ICVMs takes about 20 minutes during which time a fair few messages
are written to standard error. 
When orchestrator has finished, a set of messages should be written to stdout indicating success and which give the IP 
addresses and URLs for various things in the newly created environment.  
The following is a sample of these messages (date, time, and system indentification information has bee excluded for brevity):

\small\begin{verbatim}
   Your inception cloud 'scooter0' is ready!!!
   Gateway IP is 135.207.223.158
   Chef server WebUI is http://192.168.254.28:4040
   OpenStack dashboard is https://192.168.254.29
\end{verbatim}\normalsize

\subsection{Stopping The Inception Cloud}
The inception cloud can be stopped manually by halting all of the ICVMs, or orchestrator can be used
giving it the \verb!--cleanup! command line flag which causes it to terminate all of the ICVMs.

\small\begin{verbatim}
   orchestrator  -p scooter0  --cleanup 
\end{verbatim}\normalsize

\subsection{Finalisation}
It will take approximately 20 minutes for orchestrator to start the inception cloud.
Once orchestrator reports that the inception cloud is ready, a small amount of housekeeping should be done. 
These tasks include:

\begin{itemize}
\item Determining the network addresses of the gateway ICVM.

\item Repointing sshuttle to use the gateway ICVM and to reference the ICVM control network

\item Stopping the boot-up VM

\item Adding the controller to /etc/hosts

\item Creating credentials for your inception cloud
\end{itemize}

\subsubsection{Repointing Sshuttle}
Once the ICVMs are running, sshuttle should be "pointed" at the gateway ICVM so that the boot-up VM can be stopped. 
Sshuttle must also be set to tunnel requests for the private control network that is used by the 
ICVMs as this is the network on which the nova authorisation and dashboard processes listen on. 
The following commands illustrate how this can be done:  

\small\begin{verbatim}
   nova list | grep scooter0-gateway          
   ssh ubuntu@yyy.yyy.yyy.yyy ifconfig eth1   
   sshuttle -e ssh -A -i ~/.vmkeys/agave.shared.pk -v \
            -r ubuntu@yyy.yyy.yyy.yyy  \
            192.168.254.0/24 zzz.zzz.zzz.0/24

\end{verbatim}\normalsize
\noindent
Where:

\dlbeg{0.9in}
\dlitem{scooter0}{ Is the prefix that was given to orchestrator when the inception cloud was started.}

\vspace{5pt}
\dlitem{yyy.yyy.yyy.yyy}{ Is the public IP address for the gateway.}

\vspace{5pt}
\dlitem{zzz.zzz.zzz.0} { 
	Is the network address of the control network. The netmask also must be checked to determine if /24 is 
	the appropriate number of bits being used to represent a host id; if not it must be changed to match the netmask.  
}

\vspace{5pt}
\dlitem{ubuntu}{ Is the user name that was injected onto each of the ICVMs. }
\dlend

After these commands are executed sshuttle will again be running on the workstation and managing a tunnel between the 
workstation and both the virtual network and the inception cloud's private control network in the OpenStack environment. 

\subsubsection{Modifying /etc/hosts}
In order to use the inception cloud OpenStack dashboard (URL given in the last set of messages generated by orchestrator),
the workstation must be able to resolve the 
controller host name (e.g. \verb!scooter0-controller! using the earlier example prefix).  
The easiest way to do this is to have sshuttle forward all DNS requests to the inception cloud environment for resolution. 
This is done with the addition of a command line flag on the sshuttle command, however shuffling all of the workstation's DNS
traffic into the VM environment is probably not a very wise choice. 
Instead, the host name of the controller, and it's control network IP address (zzz.zzz.zzz.hhh) should be added to the 
\verb!/etc/hosts! file on the workstation.   


\subsubsection{Setting Credentials}
Credentials must be set in the environment to allow nova to be used on the workstation to control the iVMs in the 
inception cloud.  
The following is a list of variables that must be exported and their approximate values (the IP address supplied for the 
authorisation URL will be different as might the username).  
The password was given to the user \emph{demo} via the dashboard using the \emph{admin} user ID.

\small\begin{verbatim}
  export OS_AUTH_URL=http://10.251.0.3:5000/v2.0/
  export OS_TENANT_NAME="demo"
  export OS_USERNAME=demo
  export OS_PASSWORD=demo
\end{verbatim}\normalsize

The network address given is that of the private control network. 
The dashboard can be used to setup any users and\/or projects (tenants) that are needed in the inception cloud. 
The admin user ID and password are admin/admin by default. 

\begin{figure}
  \centering
  \includegraphics[width=\linewidth]{inception_final.eps}
  \caption{
	\small The virtual environment after cleanup showing path of browser traffic with the dashboard.
  	\label{fg_final}\normalsize
  }
  \normalsize
\end{figure}
Following housekeeping, the environment should be as shown in figure \ref{fg_final}.
