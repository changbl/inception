\section{Preparation}
To prepare the workstation and the virtual environment to start an inception cloud, the tasks listed below, and
explained in more detail in subsequent sections,  must first be performed.
All of these should be done on the workstation or from the workstation browser via the OpenStack dashboard or the
Nova client CLI.

\begin{enumerate}
\item Install software
\item Set environment variables
\item Create keys
\item Start small boot-up VM
\item Create Images
\item Add floating IP to the VM
\item Start \verb!sshuttle!
\end{enumerate}

\subsection{Install Software}
Some or all of the required software might already be installed.
Verify that the correct versions of each of the software packages are available on the workstation
and take steps to upgrade or load the missing packages as is needed.
The flavour of Linux installed on your workstation will dictate the exact commands (e.g. apt-get or zypper)
that are needed to load and\/or upgrade Python, sshuttle, and pip.
Pip can then be used to install Nova and Oslo.
Examples of each of the commands that might be needed to manage the required software packages are presented
in Appendix A.

\subsubsection{Inception source}
The source for inception is available from github.
The command below will fetch the inception source and place it in a directory under the current working directory.

\small\begin{verbatim}
   git clone https://github.com/stackforge/inception.git
\end{verbatim}\normalsize

Following the execution of the \verb!git! command, switch to the inception directory and verify that the
directory was populated.
Inception may be installed, or used from this directory.
If the decision is made to install inception, the following command should be used:

\small\begin{verbatim}
   python setup.py install
\end{verbatim}\normalsize

\subsection{Set Environment Variables}
Ensure that the environment variables which define the authorisation URL and credentials for OpenStack are set and exported.
OpenStack credentials can be obtained using the OpenStack dashboard interface and following these steps:
\label{set_env_sect}
%
\begin{enumerate}
\item Log into the dashboard
\item Click the \verb!Settings! link (top right)
\item Click the \verb!OpenStack API! link on the left
\item Click the \verb!Download RC File! button and save the file to disk.
\end{enumerate}
%&uindent
%
Once the file has been saved to disk you can source the file (the assumption is made that the
shell being used is bash compatible).
Sourcing the file should prompt for a password, and then export the following variables to the environment:


\dlbeg{1.2in}
\dlitem{OS\_AUTH\_URL}{ The reference to a process which provides user authentication.}

\dlitem{OS\_PASSWORD}{ The password for your account (you'll be prompted to enter this when the file is sourced).}

\dlitem{OS\_TENANT\_ID}{ The ID string for the tenant (currently refered to as the project on the dashboard).}

\dlitem{OS\_TENANT\_NAME}{ The human form of the tenant ID.}

\dlitem{OS\_USERNAME}{ Your user name.}
\dlend

\subsection{Create Keys}
Create (if needed) a public/private key pair and register it with OpenStack.   If you do not have a key pair, generate one using
nova. (I prefer to name these with the OpenStack cluster/environment name, and then the key name with an extension that
indicates private key: agave.scooter.pk.)

\small\begin{verbatim}
   touch agave.scooter.pk
   chmod 600 agave.scooter.pk
   nova keypair-add scooter >>agave.scooter.pk
\end{verbatim}\normalsize

The commands above will create the key, write it to disk and register the public key with OpenStack.
Executing the \verb!touch! and \verb!chmod! commands prior to generating the key adds a bit of security which prevents the exposure of the key
which results if the permissions on the key file are changed after it is generated.
Regardless of when the permissions are changed, they will need to be changed in order for the file to recognised and used by ssh.

If you already have a private key (one that several users might share) then a public key can be generated from the private key and registered with OpenStack:

\small\begin{verbatim}
   ssh-keygen -f agave.shared.pk -y >agave.shared.puk
   nova keypair-add --pub-key agave.shared.puk shared

\end{verbatim}\normalsize

If you have both a public and private key file, then the \verb!ssh-keygen! command can be skipped; it is only necessary to supply the existing public key to
OpenStack using the nova command line.

\subsection{Start Boot-up VM}
Create and initialise a tiny\footnote{If you are going to need to create OS images during setup, use a meidum VM.}
VM that will act as the initial gateway to the virtual environment for processes running on the workstation.
For the examples used in the remainder of this document, the boot-up VM is given the name \verb!scooter_bv.!
The VM should be started with the key that was registered with OpenStack.

\small\begin{verbatim}
   nova boot --image centos --flavor m1.tiny --key_name shared \
       --security_groups default scooter_bv
\end{verbatim}\normalsize

\noindent
The VM should boot quickly and once it is active you may continue.

\subsection{Create O/S Images}
If suitable inception cloud images are not already available in the current virtual environment, they must be created.
The following lists the steps necessary to create the O/S images:

\begin{enumerate}
	\item Ssh to the boot-up VM.
	\item Copy inception utility scripts from the source installed on the workstation.
	\item Run the script \verb!pre_switch_kernel.sh! to convert from a vertual kernel to a generic kernel.
	\item Wait for the VM to reboot and log in again.
	\item Run the script \verb!pre_install_ovs.sh! to install open vswitch.
	\item Using nova, or the dashboard, take a snapshot of the VM giving it an image name of XXX-gv.
	\item Run the script \verb!pre_install_chefserver.sh! to install the Chef software.
	\item Create another image naming it XXX-gvc.
\end{enumerate}

Image names (shown as XXX in the above list), can be anything that aligns with local policy and are used
with the \verb!--image! and \verb!--chefserver_image! options on the \verb!orcehestrator! command line.

\subsection{Add A Floating IP Address}
Add a floating (public) IP address to the new VM so that it can be reached from the "real world."
(xxx.xxx.xxx.xxx is one of the public floating IP addresses that are available; use \verb!nova floating-ip-list!
to get a list of available addresses.

\small\begin{verbatim}
   nova  add-floating-ip scooter_bv xxx.xxx.xxx.xxx
\end{verbatim}\normalsize

\subsection{Start sshuttle}
The sshuttle programme creates a tunnel through ssh allowing programmes on the workstation to access VMs created on the
same internal network as the boot-up VM without having to assign each VM a public address.
The sshuttle command is given the private key portion of the key pair that was used to start the boot-up VM; this is
necessary to allow sshuttle to start an ssh session through which the tunnel is created.
The user name (ubuntu in the example below) is any user on the boot-up VM that is available and allows access via
the key (the assumption is that OpenStack created this user, or it was a part of the saved imagee,  and the public key was
inserted into the \verb!authorized_keys! file in the user's \verb!.ssh! directory.)

\small\begin{verbatim}
   sshuttle -e ssh -A -i ~/.vmkeys/agave.shared.pk -v \
      -r ubuntu@xxx.xxx.xxx.xxx 192.168.254.0/24
\end{verbatim}\normalsize

\noindent
The \verb!-v! option causes sshuttle to be more verbose with messages to the standard error device.
Ultimately, redirecting the output of sshuttle to \verb!/dev/null! and running the process asynchronously, is probably
wise, but initially seeing the verbose messages scroll by in the window is a nice confirmation that the tunnel is active and data
is being transferred.
The \verb!xxx! IP address is the public floating IP address assigned earlier.  The second address (192.168.254.0 in the example)
bb the virtual network that is created by OpenStack.
If it is not known, the following command (with suitable path for the public key) might provide the address:

\small\begin{verbatim}
   ssh -i ~/.vmkeys/agave.shared.pk ubuntu@xxx.xxx.xxx.xxx ifconfig eth0
\end{verbatim}\normalsize

The netmask can be used to determine the number of bits of the address that are treated as the network ID (probably 24) and thus
added after the slant on the sshuttle command.

\vspace{20 pt}
